\documentclass[UTF8]{book}
\usepackage{graphicx}
\usepackage{caption}
\usepackage{subcaption}
\usepackage{float}
\usepackage{amsmath}
\usepackage{seqsplit}
\usepackage{tikz}
\usepackage{pgfplots}
\usepackage{listings}
\usepackage{CJK}

\newtheorem{theorem}{Proposition}
\newcommand\longnumber[2]{%
    \begin{minipage}{#1}
    \seqsplit{#2}
    \end{minipage}
    }
\newcommand*\thickdash{\includegraphics{thick-dash2}}
\newcommand*\thickdot{\includegraphics{thick-dot2}}

\graphicspath{ {images/} }
\begin{document}
\begin{CJK}{UTF8}{gbsn}

\title{The Network}
\author{Dan Auerbach}
\date{2015}
\maketitle

\part{The Telephone}

In 1876, to celebrate the centennial of the Declaration of Independence, the United states hosted its first World's Fair in Philadelphia. The event lasted several months, thirty-seven countries participated, and nearly 10 million visitors attended.

The biggest draws were primarily mechanical contraptions: displays of steam engines, the world's Remington Typographic Machine (typewriter), precision watches, to name a few. Yet electrical inventions also drew crowds, and tucked away in a small area across the Machinery Hall sat a young Alexander Graham Bell, demonstrating the transfer of human voice through wires.

People viewed the early telephone as a fun toy, not a serious technological contender to the business-oriented telegraph empire. These early telephone prototypes were limited in their range, typically just connecting two adjacent rooms, and it was not obvious at first that the technology could be improved upon to rival the telegraph network. Yet it did not take long for commercial telephony to develop, and for power to shift away from Western Union and instead towards AT\&T, a corporation that would define the telecommunications industry in the United States for over a century.

In this part, we will examine how the telephone works: what is sound, how an analog signal transmitted over a wire, what infrastructure allowed AT\&T provided telephone access to the vast majority of Americans in their home, and how the regulatory landscape began to evolve as the telecommunications industry matured.

\chapter{Signals}

\section{Sound}

When a tree falls and hits the ground, some of the energy upon impact is transferred to nearby air particles, pushing them outwards as a wave.

[IMAGE]

The wave is a \emph{longitudinal} wave, meaning it does not have the peaks and troughs of an ocean wave perpendicular to the direction of travel, but instead expands and contracts along a single dimension as it travels. Imagine a nuclear explosion, except that instead of an outwardly expanding sphere of fire, there is an outwardly expanding sphere where air particles are more tightly bunched together momentarily. The bunching of particles constitutes a change in \emph{air pressure}.

As objects on Earth interact with each other on land or in the air, these waves made up of changes in air pressure are created. Since objects interact with each other constantly, there are constant changes in air pressure. These air pressure changes signal important events, like when something might eat you or when you might be able to eat something. And so many species including us evolved ears to make sense of the constantly changing air pressure; we call our experience of it \emph{hearing}. In parallel to our sense of hearing, we evolved increasingly sophisticated ways to make sound ourselves, culminating in vocal chords and speaking.

The simplified description above might lead you to believe that there is only a single and uniform experience of a sound, but of course that's not the case. Sound has many different qualities, like pitch, intensity, duration, and timbre. It is easy to tell the singing of a friend from the sound of a tree falling in the woods. This is possible because sound waves can be rather sophisticated, the air particles bunching and unbunching in subtle non-uniform ways.

Here is a representation of the sound of a tree falling:

[IMAGE]

And here is a representation of human speech:

[IMAGE]

The above representations are called \emph{time domain representations}: the x axis represents time, and the y axis represents the instantaneous change in air pressure. Large values on the y axis mean that particles are tightly compressed at that instant. Of course these representations have to be taken from some particular point in space: two people standing in different places may have different auditory experiences of the same tree fall event.

\section{Functions}

The two figures above representing various sounds are \emph{functions}. Functions constitute a central concept in mathematics.

A function is a mapping from one set of objects to another:

[IMAGE]

We say that a function has a \emph{domain}, which is the set of objects on the left in the figure above, and a \emph{codomain} which is the set of objects on the right.





\section{Basics of signals}


\section{Signals on a wire}




\chapter{The Access Network}

asdfsadf


\chapter{garbage}

The sense of hearing and the ability to make intricate sounds through our vocal system has special significance to the history of communication, since it allowed humans to begin communicating via spoken language.

\end{CJK}
\end{document}
