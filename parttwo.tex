\documentclass[UTF8]{book}
\usepackage{graphicx}
\usepackage{caption}
\usepackage{subcaption}
\usepackage{float}
\usepackage{amsmath}
\usepackage{amsfonts}
\usepackage{seqsplit}
\usepackage{tikz}
\usepackage{pgfplots}
\usepackage{listings}
\usepackage{CJK}

\newtheorem{theorem}{Proposition}
\newtheorem{definition}{Definition}
\newcommand\longnumber[2]{%
    \begin{minipage}{#1}
    \seqsplit{#2}
    \end{minipage}
    }
\newcommand*\thickdash{\includegraphics{thick-dash2}}
\newcommand*\thickdot{\includegraphics{thick-dot2}}

\graphicspath{ {images/} }
\begin{document}
\begin{CJK}{UTF8}{gbsn}

\title{The Network}
\author{Dan Auerbach}
\date{2015}
\maketitle

\part{The Telephone}

In 1876, to celebrate the centennial of the Declaration of Independence, the United states hosted its first World's Fair in Philadelphia. The event lasted several months, thirty-seven countries participated, and nearly 10 million visitors attended.

The biggest draws could safely be said to be mechanical contraptions: displays of steam engines, the world's Remington Typographic Machine (typewriter), and precision watches, to name a few. Yet electrical inventions were also on display, and tucked away in a small area across the Machinery Hall sat a young Alexander Graham Bell, demonstrating the transfer of human voice through wires.

People viewed early telephone prototypes as a fun toy, not a serious technological contender to the business-oriented telegraph empire. That made sense since these prototypes were limited in their range, typically just connecting two adjacent rooms, and it was not obvious at first that the technology could be improved upon to rival the telegraph network. Yet it did not take long for commercial telephony to develop, and once that happened power shifted away from the well-established Western Union and instead towards AT\&T, a then start-up corporation that would come to define the telecommunications industry in the United States for over a century.

In this part of the book, we will examine how the telephone works: what is sound, how is an analog signal transmitted over a wire, what infrastructure allowed AT\&T provided telephone access to the vast majority of Americans in their home, and how the regulatory landscape began to evolve as the telecommunications industry matured.

\chapter{Signals}

\section{Sound}

When a tree falls and hits the ground, some of the energy upon impact is transferred to nearby air particles, pushing them outwards as a wave.

[IMAGE]

The wave is a \emph{longitudinal} wave, meaning it does not have the peaks and troughs of an ocean wave perpendicular to the direction of travel, but instead expands and contracts along a single dimension as it travels.

[IMAGE]

Imagine a nuclear explosion, except that instead of an outwardly expanding sphere of fire, there is an outwardly expanding sphere where air particles are more tightly bunched together for a split second. The momentary compression of air particles constitutes a change in \emph{air pressure} in that location.

As objects on Earth interact with each other, waves of pressure expand outward in the surrounding medium (e.g. the air). These air pressure changes signal important events, like when something might eat you or when you might be able to eat something. Given the importance of this information for survival, many species including humans evolved auditory systems to make sense of the constantly changing air pressure; we call our experience of it \emph{hearing}. In parallel to the development of our sense of hearing, we evolved increasingly sophisticated ways to create sound ourselves, culminating in vocal chords and speaking.

The simplified description above of a tree falling might lead one to believe that there is only a single and uniform experience of a sound, but of course that's not the case. Sound has many different qualities, like pitch, intensity, duration, and timbre. It is easy to tell the singing of a friend from the sound of a tree falling. This is possible because sound waves can be rather sophisticated, the air particles compressing and un-compressing in subtle, non-uniform ways.

Here is a representation of the sound of a tree falling:

[IMAGE]

And here is a representation of human speech:

[IMAGE]

The above figures are called \emph{time domain representations}: the x axis is time, and the y axis represents the instantaneous change in air pressure. Large values on the y axis mean that particles are tightly compressed at that instant. Of course these representations have to be taken from some particular point in space: two people standing in different places may have different auditory experiences of the same tree-falling event.

\section{Functions}

The two figures above representing various sounds are examples \emph{functions}, a central concept in mathematics.

A function is a mapping from one set of objects to another:

[IMAGE]

We say that a function has a \emph{domain}, which is the set of objects on the left in the figure above, and a \emph{co-domain} which is the set of objects on the right. There are a couple of important rules that must be followed for something to be called a function:

\begin{itemize}
\item Functions must be defined for ALL elements of the domain
\item Functions must have EXACTLY ONE value in the co-domain for each element of the domain
\end{itemize}

In other words, this isn't allowed:

[IMAGE OF FUNCTION NOT DEFINED ON DOMAIN]

And neither is this:

[IMAGE OF FUNCTION THAT IS MULTI-VALUED]

Note that the second rule above does not apply to the co-domain, meaning it is perfectly acceptable for several values in the domain to be mapped to a single value in the co-domain:

[IMAGE OF NON-1-1 FUNCTION]

There are no restrictions on the domain or co-domain. You can have a function where the domain is pet owners on Earth, and the co-domain is the set of pets:

[IMAGE]

We might call this function ``FavoritePet'', and write $FavoritePet(John) = Fluffy$ to mean that the function ``FavoritePet'' maps the element ``John'' in the domain of pet owners to the element ``Fluffy'' in the domain of pets. (Exercise: would `Owns'' with the same domain and co-domain constitute a function?)

To indicate that a function $f$ has domain $A$ and co-domain $B$, we write $f: A \rightarrow B$.

The domain and co-domain of a function can be finite (like Pet Owners) or infinite (like natural numbers: $\mathbb{N} = {0, 1, 2, 3, etc.}$). The domain and co-domain could be the same, but they do not have to be.

Sometimes functions can be described easily, like ``FavoritePet'' or the function $f: \mathbb{N} \rightarrow \mathbb{N}$ that takes a natural number and doubles it. We could call this function ``Doubles'', but mathematicians prefer a more compact and precise description: $f(x) = 2x$. Other times functions cannot be described easily at all: they are simply a random looking mapping from the domain to the co-domain. In this case, the only way to give a full description of the function is to give it's value for every single input object, a task that is sometimes provably impossible when there are an infinite number of input objects.

Domains or co-domains might be multi-dimensional spaces. For example, a function $f: \mathbb{N}^{3} \rightarrow \mathbb{N}$ maps triplets like $(4,5,1)$ to natural numbers like $17774$. Domains and co-domains can also be discrete, like the natural numbers. Or they can be continuous, like pitch, or time.

A function is an abstract object, and as such it has many representations. For example, the function $f(x)=2x$ can also be described by plotting it on a two dimensional axis, where the x-axis is the input of the function and the y-axis is the output:

[PLOT OF f(x)=2x]

It could also be represented with the primary visualization that we used to introduce the notion of a function:

[IMAGE OF DOTS]

We will return to functions all the time, so it is important to develop a sharp understanding of functions.

\emph{Exercise}: write out a mathematical description for a function $f$ that takes two numbers, and raises the first to the power of the second, then subtracts the second. What is the value of $f(3,4)$? As a bonus, use graphing software to create a graphical representation of this function as a surface.

\section{Speech on a wire: an analog approach}

Above, we saw examples of functions that represented sound. Here again is our time domain representation of speech, also called a speech \emph{waveform}:

[IMAGE of speech]

This is just a function in which the domain is time and the co-domain is air pressure. The information in the figure above is enough to create a sound that we would recognize as speech. In order for telephony to be possible, we need a way to continuously transmit the information of the sound waveform from one location to another.

An electrical wire does not have air particles that can be compressed and uncompressed, but it does have other properties that can be varied continuously over time. In particular, the current flowing through a wire can increase and decrease. This is how the telegraph worked except that in that case, the current would be on or off, resulting in a function that might look like this:

[IMAGE OF TELEGRAPH CURRENT SIGNAL]

The telephone has to solve the more subtle problem of receiving the human speech form, reproducing the information is contains by varying electrical current, and then reproducing the sound at the other end:

[CHART OF CONCEPTUAL OUTLINE OF TELEPHONE]

\section{Basics of signals}

We will see how the telephone accomplishes this shortly, but first let us turn our attention to the idea of a \emph{signal}.

We have already covered symbolic systems, and the fact that any symbolic representation of information is interchangeable [ADD NAME FOR THIS THEOREM?]. But what happens when the information we care about is something like the human voice that cannot be easily represented as a set of symbols? Recall that in this scenario, we say that the information is \emph{analog}.

There is no single, neat definition of a signal. For our purposes, however, let us use the following slightly narrow and restrictive definition:

[TODO FIX THIS DEFINITION]

\begin{definition}
A \emph{signal} is a function where the domain is time.
\end{definition}

\section{Information content of an analog signal}

How much information does a sound signal contain?

\chapter{How A Telephone Works}

The telephone transforms a sound signal into an electrical signal and back again. It was not the only device in the late 1800s that was seeking to gain mastery over sound.

The phonograph was invented in 1877 and was the original technology in a line that continued with record players, cassette tapes players, CD players, and MP3 players. [FOOTNOTE: it seems this line of technology may have been subsumed by more general purpose desktop and mobile computers in the last 5 years or so]. The phonograph was aimed at solving a similar problem to the telephone, in that it needed to capture a sound signal somehow. The difference is that once captured, instead of the moving the sound signal quickly over a long distance, the phonograph instead was charged with creating a physical object from which the sound could be later retrieved, a record. Early phonographs used a circular disk as the record, and a sound signal was captured by deviations in a spiral groove. [MORE HERE?]

[DIAGRAM OF TELEPHONE VS PHONOGRAPH]

\section{A basic telephone}

A one-way telephone has a microphone at one end capable of translating an analog sound signal into an electrical one, and a receiver at the other end capable of translating the electrical signal back into an audible sound signal:

[DIAGRAM]

In this diagram, the electrical circuit is always closed, meaning that current is flowing continuously through the circuit without any means to stop it (aside from cutting the cord, perhaps). In practice, we will want a mechanism to open the circuit and cut off the flow of current, which helps to save battery. We also will want a telephone to work in the familiar two-way fashion in which you can talk and listen at the same time.

\section{Microphone}

A \emph{microphone}




\chapter{The Telephone Network}



\end{CJK}
\end{document}
