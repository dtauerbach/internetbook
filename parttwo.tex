\documentclass[UTF8]{book}
\usepackage{graphicx}
\usepackage{caption}
\usepackage{subcaption}
\usepackage{float}
\usepackage{amsmath}
\usepackage{seqsplit}
\usepackage{tikz}
\usepackage{pgfplots}
\usepackage{listings}
\usepackage{CJK}

\newtheorem{theorem}{Proposition}
\newcommand\longnumber[2]{%
    \begin{minipage}{#1}
    \seqsplit{#2}
    \end{minipage}
    }
\newcommand*\thickdash{\includegraphics{thick-dash2}}
\newcommand*\thickdot{\includegraphics{thick-dot2}}

\graphicspath{ {images/} }
\begin{document}
\begin{CJK}{UTF8}{gbsn}

\title{The Network}
\author{Dan Auerbach}
\date{2015}
\maketitle

\part{The Telephone}

In 1876, to celebrate the centennial of the Declaration of Independence, the United states hosted its first World's Fair in Philadelphia. The event lasted several months, thirty-seven countries participated, and nearly 10 million visitors attended.

The biggest draws were primarily mechanical contraptions: displays of steam engines, the world's Remington Typographic Machine (typewriter), precision watches, to name a few. Yet electrical inventions also drew crowds, and tucked away in a small area across the Machinery Hall sat a young Alexander Graham Bell, demonstrating the transfer of human voice through wires.

People viewed the early telephone as a fun toy, not a serious technological contender to the business-oriented telegraph empire. These early telephone prototypes were limited in their range, typically just connecting two adjacent rooms, and it was not obvious at first that the technology could be improved upon to rival the telegraph network. Yet it did not take long for commercial telephony to develop, and for power to shift away from Western Union and instead towards AT&T, a corporation that would define the telecommunications industry in the United States for over a century.

In this part, we will examine how the telephone works: what is sound, how an analog signal transmitted over a wire, how AT&T provided telephone access to the vast majority of Americans in their home, and how the regulatory landscape began to evolve as the telecommunications industry matured.

\chapter{Sound}

When a tree falls and hits the ground, some of the energy upon impact is transferred to nearby air particles, pushing them outwards as a wave.

[IMAGE]

The wave is a \emph{longitudinal} wave, meaning it does not have the peaks and troughs of an ocean wave as it travels, but instead expands and contracts along a single dimension. Imagine a nuclear explosion, except that instead of an outwardly expanding sphere of fire, there is an outwardly expanding sphere where air particles are more tightly bunched together momentarily. The bunching of particles constitutes a change in \emph{air pressure}.

As objects on Earth interact with each other, these air pressure waves are created. Since objects interact with each other a lot, our world is one in which there are constant changes in air pressure that signal important events, like when something might eat you or when you might be able to eat something. So we evolved ears to make sense of the constantly changing air pressure, and we call our experience of it \emph{hearing}. This sense has special significance to the history of information, since it allowed humans to begin communicating in sophisticated ways.








A split second after impact, the particles are bunched together a few feet from the impact and over time, the frontier of bunched particles expands.



Air particles surrounding the tree are pushed outward, so that there is a bunch of them like this:

[IMAGE]





It may sound strange to say that sound is a mechanical wave, but that's because

Given a world awash in small constant pressure changes.

Our world is made up of molecules.



blah blah blah

\chapter{The Access Network}

asdfsadf

\end{CJK}
\end{document}
