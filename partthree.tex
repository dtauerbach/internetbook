\documentclass[UTF8]{book}
\usepackage{graphicx}
\usepackage{caption}
\usepackage{subcaption}
\usepackage{float}
\usepackage{amsmath}
\usepackage{amsfonts}
\usepackage{seqsplit}
\usepackage{tikz}
\usepackage{pgfplots}
\usepackage{listings}
\usepackage{CJK}

\newtheorem{theorem}{Proposition}
\newtheorem{definition}{Definition}
\newcommand\longnumber[2]{%
    \begin{minipage}{#1}
    \seqsplit{#2}
    \end{minipage}
    }
\newcommand*\thickdash{\includegraphics{thick-dash2}}
\newcommand*\thickdot{\includegraphics{thick-dot2}}

\graphicspath{ {images/} }
\begin{document}
\begin{CJK}{UTF8}{gbsn}

\title{The Network}
\author{Dan Auerbach}
\date{2015}
\maketitle

\part{Radio}

The idea of invisible radio waves conveying information is at once commonplace and magical. Since we use phones, tablets, laptops, and other computers of different shapes and sizes every day, people have absorbed the idea that a mobile device without the help of any wires can communicate with a cell tower or a home WiFi router. But it is magical because unless you are an electrical engineer or have a background in radio, you probably have no idea how the stream of 1s and 0s that encode a cat video actually make it from the cell tower to your phone. It's an interesting psychological phenomenon. We accept as commonplace communication infrastructure that seems utterly fantastical, trusting others with science and engineering backgrounds to have worked out all the details, and this miraculous achievement becomes dull to all but the most devoted readers interested in peeking under the hood.

Our first project, then, is to demystify radio, so that it will become clear exactly how information can be transmitted without wires by making use of the electromagnetic spectrum.

Next, we will discuss the development of radio within the global information network that was developing in the early and mid 20th century. The development of radio technology was critical first and foremost because it was a demonstration that we don't always need wires everywhere but could instead rely on radio links to carry information. But moreover, just as the technology of radio was proving itself out for two-way communication, the ascendance of broadcast radio in the early 20th century established a the dominant paradigm for entertainment of the last century: a single broadcasting entity could attract listeners and followers by the millions. You just had to be lucky enough to have an in with someone controlling the microphone.

\chapter{Signals in light}

In the last part[TODO: specify] of the book, we discussed signals, or information-bearing functions where the domain is time.

We discussed in particular how a sound wave could be abstracted away to its signal form, transposed into electrical current running through wires with the help of some electromechanical components, and finally transposed back again into sound.

But the idea of a signal is abstract, and doesn't necessitate the the use of electricity. So what if we choose a different medium to temporarily translate our sound waves in our thought experiment. Not electricity, but light. Our earliest thought experiment with Mary and [TODO fill in] demonstrated how \emph{symbolic} information could be transmitted via light, so let's return to that example with an eye towards transmitting not a discrete set of symbols, but the continuous peaks and troughs of an analog signal.

To do this, we must discard the simplifying assumption that a light is on or off. It's an easy assumption to part with, as we all know that a given light can be in a continuous number of states of brightness, and the level of brightness can vary with time -- just think of the dimmer controlling the lights of your dining room, and imagine turning it slowly and then quickly and plotting the brightness of the resulting light on a $y$ axis over time. That's our signal, in light form.

[TODO: image?]

\chapter{Electromagnetic spectrum}

So how might you go about actually transmitting a signal via light? We might imagine hooking up our microphone speaker to our dining room dimmer. This wouldn't be too hard, but we would need something on the receiving end to be able to read the level of brighness at any given moment. Later, we'll discuss fiber optic cables which do use light to transmit both symbolic information and analog signals.

For now, let's set aside the exercise of inventing some practical contraption which would allow us to use light in place of electric wires. Instead, armed with the core notion that a signal can exist as a variation in the brightness of light.





\end{CJK}
\end{document}
